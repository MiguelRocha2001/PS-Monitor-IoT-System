\documentclass[a4paper,twoside,11pt]{article}
\usepackage[utf8]{inputenc}
\usepackage[portuguese]{babel}
\usepackage{graphicx}
\usepackage{url}
\usepackage{xcolor}

% pdflatex

% redefinição das margens das páginas
\setlength{\textheight}{24.00cm}
\setlength{\textwidth}{15.50cm}
\setlength{\topmargin}{0.35cm}
\setlength{\headheight}{0cm}
\setlength{\headsep}{0cm}
\setlength{\oddsidemargin}{0.25cm}
\setlength{\evensidemargin}{0.25cm}

\begin{document}


\begin{figure}
    \
        \centering
        \includegraphics[scale = 0.4]{images/logoISEL.png}
%        \begin{subfigure}{.6\textwidth}
%        \centering
%        \includegraphics[scale = 0.15]{athens.jpg}
%        \end{subfigure}
\end{figure}

\title{\Large\textbf{\texorpdfstring{%
    {Project Proposal for Project and Seminary 2022/23}\\[0.5ex]\rule{\linewidth}{0.5pt}\\[0.5ex]
    IoT System for pH Monitoring in Manufacturing Facilities}%
}{}}

\author{
\begin{tabular}{c}
             Miguel Rocha, n.º 47185, e-mail: a47185@alunos.isel.pt, tel.: 933740977\\
             Pedro Silva, n.º 47128, e-mail: 47128@alunos.isel.pt, tel.: 935122223\\
\end{tabular}}

\date{
\begin{tabular}{ll}
  {Supervisor:} & Rui Duarte, e-mail: rui.duarte@isel.pt\\
\end{tabular}\\
\vspace{5mm}
March 20, 2023}

% \begin{document}

\maketitle

\section{Introduction}
The Internet of Things (IoT) is revolutionizing the way machines work and communicate with people.
Our main objective consists in the creation of a smart and automated system, using IoT technology to enhance the efficiency and accuracy of the machine filter process.
By incorporating sensors into these machines, we can detect problems before they lead to costly downtime or safety hazards.
The sensors will collect data relative to the machine's environment, such as the water temperature and pH.
All this data is sent to a central server, and through real-time data analysis, we identify if the machine is not working optimally or is closer to failure.
In these situations, an alert will be sent to a device controlled by the machine operator or manager, enabling them to take prompt action to resolve the issue.
Our project seeks to improve efficiency, reduce costs and increase safety by automating the fault detection process of machine inspection.

\section{Requirements}
\begin{itemize}
    \item Economic hardware
    \item High battery life device
    \item Low false alarm frequency
    \item Visual and auditory alarm
    \item Hardware documentation
\end{itemize}
\section{Objectives}
\subsection{Main routes}
 \begin{itemize}
    \item Programming the MCU to collect and send data
    \item Setting the message broker
    \item Implementing the back-end service to manage system logic (data analysis and alert system)
    \item Implementing a user-friendly interface to interact with the system (visualize data, adding devices, etc)
    \item Implementing the database to store system data
\end{itemize}
\subsection{Optional routes}
    One possibility to enhance the functionality of the system is to contemplate the inclusion of supplementary sensors, such as those for measuring humidity and ambient temperature. Another route could be to develop an Android app as an additional option for users, providing an alternative to the web-based application.

\section{Justification}

The development of this project will provide us with valuable experience in one of the fastest growing areas of technology in the world -Internet of Things (IoT).
By utilizing an industrial setting as inspiration, we aim to develop new skills, commencing with the programming of an MCU which is the core of IoT contexts.

We will also learn alternative network protocols, mainly used in IoT systems. By doing so, we hope to gain a deeper understanding of how these protocols work and their benefits compared to traditional HTTP protocols.

The implementation of this project will provide us with an opportunity to further our understanding of various topics covered during our academic course. These include web programming, database design and usage, and the design architecture of small to medium scale software projects. By applying these concepts in a practical setting, we will gain a deeper knowledge and appreciation for their importance and relevance in the real world. 


\section{Scope}
The functional scope of the project is limited to the collection and analysis of relevant data, related to the target device, and the interaction with the system to configure preferences and to obtain collected data, namely:
\begin{itemize}
    \item Associate/remove IoT devices to/from the IoT platform
    \item Having the ability to visualize collected data, in a graph form
    \item Notification systems to alert the admin/manager of a possible machine filter malfunction
    \item System thresholds configuration (PH or/and water temperature critical level)
\end{itemize}

\section{Approach and Deliverables}
\subsection{Approach}

The first module will be composed by an MCU and it's sensors. For this, we will use an ESP32-S2 development kit, which supports all technologies and peripherals necessary for the success of our system, including Wi-Fi, built-in security, and very good community support, despite being a relatively inexpensive board. It will involve integrating a pH sensor and water temperature sensor into a small 3.6V battery-powered board. To program the board, we will utilize the C language, along with the ESP-IDF framework, which offers robust hardware control capabilities, specific to this board family.

The Broker makes up the second module. We will either implement the Broker ourselves or evaluate the potential benefits of subscribing to a third-party Broker based on the project's scale.
The data that is published will be stored in a database utilizing either the PostgreSQL or MySQL engine.

We have opted to use Kotlin as the programming language for our backend server due to its modern and robust features. Additionally, we have decided to leverage the Spring framework for building our backend service since it is widely considered one of the top server-side frameworks for the Java (and thus Kotlin) programming language, considering our previous experience and proficiency in this technology.

We aim to develop a user interface either by creating a website using React and Boostrap from scratch, or by utilizing open-source software to build a web application that allows for efficient management of the sensor data.



\subsection{Deliverables}
The end solution will comprise:
\begin{itemize}
    \item A MCU (micro-controller), which will collect and transmit (collected) data from the sensors to the IoT platform
    \item An IoT platform, composed by:
    \begin{itemize}
        \item A Broker server, that receives the data, from the MCU
        \item A relational database to store system data (records, devices, etc) 
        \item A backend service to:
        \begin{itemize}
            \item Manage system logic
            \item Analyze data
            \item Send user notifications
            \item Manage user data (ex: devices)
        \end{itemize}
    \end{itemize}
    \item Unit tests for all system modules
    \item A fully functional Web App, to interact with the system (covering the functionalities defined in the scope)
\end{itemize}
´
\bigbreak

\section{Constraints and assumptions}
    Our team has limited experience in programming MCUs, and is relatively new to IoT platforms. It is also relevant that all hardware has to be possible to acquire in a timely manner due to the time bound nature of the project.
    
\section{Resources}
    For our project, we will need a suitable development board, such as the ESP32-S2, and sensors to measure environmental parameters, such as water temperature and pH. Also we plan to utilize open software which is easily available, at no additional cost, on our personal devices or university.


\section{Risks}
    \begin{itemize}
        \item Due to our limited experience with IoT platforms, we expect to encounter a significant learning curve during this project, particularly in programming the MCU and utilizing the MQTT network protocol.
        \item Like any other hardware project, unforeseen issues may arise, that are beyond our control and inherent to the hardware's nature. These issues could potentially impact our project timeline and organization.
    \end{itemize}

\section{Project Organization}
Pedro Silva and Miguel Rocha will carry out the project under the supervision of Rui Duarte. The project is sponsored by Mommertz, a German company.

\section{Major Milestones}
    \begin{itemize}
        \item Project Proposal -  March 20, 2023
        \item Progress presentation -  April 24, 2023
        \item Beta version (report, poster, and organization) -  June 5, 2023
        \item Final version (Academic Calendar 2022/2023) -  July 10, 2023 
    \end{itemize}

    
\bibliographystyle{unsrt}
\bibliography{referencias}

\end{document}